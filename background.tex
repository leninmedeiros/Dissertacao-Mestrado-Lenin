\chapter{\textit{Background}}
Este trabalho aborda, principalmente, dois conceitos importantes. O primeiro diz respeito ao tipo de site que foi investigado e ao qual se tentou propor alguma melhoria de design. O segundo se refere ao ato de coletar algum tipo de informação utilizando, como auxílio, as conexões sociais estabelecidas entre indivíduos. Tal recurso está ligado com a possível melhoria de design estudada durante este trabalho. Portanto, faz-se necessário apresentar neste capítulo: um levantamento em mais detalhes dos significados destes dois conceitos, bem como uma visão geral do que já está abordado na literatura em relação aos mesmos.

\section{Sites do Tipo \textit{Social Q\&A}}
Sites que são acessados por usuários com o intuito de realizar perguntas, responder a perguntas ou visualizar as discussões geradas por perguntas e os seus respectivos conjuntos de respostas estão conceitualizados na literatura de duas maneiras distintas: \textit{community Q\&A} [REF] e \textit{social Q\&A} [REF].

O primeiro termo supracitado é mais específico do que o segundo. Para um site ser considerado deste tipo, é preciso que haja uma identificação de indicadores formais de comunidade, como usuários se mostrando engajados em divulgá-lo, adotando e expressando uma identidade, etc. [REF]

\textit{Social Q\&A}, de acordo com [REF], é um termo mais abragente e se refere aos sites nos quais os respectivos usuários...CONTINUAR

Apesar de sites do tipo \textit{social Q\&A} serem considerados instâncias de comunidades \textit{online} [REF], o que pode remeter ao conceito de \textit{community Q\&A}, todos os trabalhos utilizados na elaboração da fundamentação teórica desta pesquisa e, portanto, relevantes neste escopo, optam por utilizar o primeiro conceito definido anteriormente como \textit{social Q\&A}. Tal opção também foi escolhida durante a elaboração desta dissertação. 
\begin{itemize}
  \item Definição;
  \item Trabalhos sobre (dando um certo destaque aos trabalhos que tentam ajudas os perguntadores a encontrar respostas/melhores respostas);
  \item O que o meu trabalho tem a contribuir em relação ao campo de social q\&a sites?
\end{itemize}
\section{\textit{Friendsourcing}}
\begin{itemize}
  \item Definição;
  \item Trabalhos sobre;
  \item O que o meu trabalho tem a contribuir em relação ao campo de friendsourcing?
\end{itemize}