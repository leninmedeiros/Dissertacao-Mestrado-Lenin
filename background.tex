\chapter{\textit{Background}}
Este trabalho aborda, principalmente, dois conceitos importantes. O primeiro diz respeito ao tipo de site que foi investigado e ao qual se tentou propor alguma melhoria de design. O segundo se refere ao ato de coletar algum tipo de informação utilizando, como auxílio, as conexões sociais estabelecidas entre indivíduos. Tal recurso está ligado com a possível melhoria de design estudada durante este trabalho. Portanto, faz-se necessário apresentar neste capítulo: um levantamento em mais detalhes dos significados destes dois conceitos, bem como uma visão geral do que já está abordado na literatura em relação aos mesmos.

\section{Sites do Tipo \textit{Social Q\&A}}
Existem sites que são acessados por usuários com o intuito de: realizar perguntas, responder a perguntas ou, ainda, visualizar as discussões geradas por perguntas e os seus respectivos conjuntos de respostas. Tais sites estão conceitualizados na literatura de duas maneiras distintas: \textit{community Q\&A} [REF] e \textit{social Q\&A} [REF].

O primeiro termo supracitado é mais específico do que o segundo: para um site ser classificado assim, é preciso que haja uma identificação de indicadores formais de comunidade, como usuários se mostrando engajados em divulgá-lo, adotando e expressando uma identidade, etc. [REF]

\textit{Social Q\&A}, de acordo com [REF], é um termo mais abragente e se refere aos sites nos quais os respectivos usuários realizam perguntas, respondem a perguntas e avaliam o conteúdo do site enquanto estão interagindo com ele.

Apesar de sites do tipo \textit{social Q\&A} serem considerados instâncias de comunidades \textit{online} [REF], o que pode remeter ao conceito de \textit{community Q\&A}, a maior parte dos trabalhos utilizados na elaboração da fundamentação teórica desta pesquisa e, portanto, relevantes neste escopo, opta por utilizar o primeiro conceito definido anteriormente como \textit{social Q\&A}. Tal opção também foi escolhida durante a elaboração desta dissertação.

O site \textit{Stack Overflow}\footnote{http://stackoverflow.com/} é um exemplo do tipo destacado acima. Tal ambiente colaborativo foi concebido em 2008 e é creditado como o maior site de perguntas e respostas voltados para programadores que desejam tirar dúvidas e oferecer ajuda [REF]. Possui aproximadamente dois milhões de contribuidores [REF] e costuma ter suas perguntas respondidas em um tempo mediano de 11 minutos [REF].

Em 2014, foi criada uma versão em português do site supracitado chamada \textit{Stack Overflow em Português}, com pouco mais de 14 mil usuários registrados [REF]. Tal site foi utilizado como estudo de caso nesta pesquisa (isto está exposto em detalhes mais adiante nesta dissertação).

COLOCAR AQUI EXEMPLOS COM FIGURAS DO MOTIVO PELO QUAL O SO PT É CONSIDERADO DO TIPO SOCIAL Q\&A

\begin{itemize}
  \item Definição;
  \item Trabalhos sobre (dando um certo destaque aos trabalhos que tentam ajudas os perguntadores a encontrar respostas/melhores respostas);
  \item O que o meu trabalho tem a contribuir em relação ao campo de social q\&a sites?
\end{itemize}
\section{\textit{Friendsourcing}}
\begin{itemize}
  \item Definição;
  \item Trabalhos sobre;
  \item O que o meu trabalho tem a contribuir em relação ao campo de friendsourcing?
  \item Não esquecer de discutir aqui a questão do capital social! (Talvez seja o caso de colocar isto como subseção)
\end{itemize}