\chapter{\textit{Background}}
Este trabalho aborda, principalmente, dois conceitos importantes. O primeiro diz respeito ao tipo de site que foi investigado e ao qual se tentou propor alguma melhoria de design. O segundo se refere ao ato de coletar algum tipo de informação utilizando, como auxílio, as conexões sociais estabelecidas entre indivíduos. Tal recurso está ligado com a possível melhoria de design estudada durante este trabalho. Portanto, faz-se necessário apresentar neste capítulo: um levantamento em mais detalhes dos significados destes dois conceitos, bem como uma visão geral do que já está abordado na literatura em relação aos mesmos.

\section{Sites \textit{Q\&A}}
% Existem sites que são acessados por usuários com o intuito de: realizar perguntas, responder a perguntas ou, ainda, visualizar as discussões geradas por perguntas e os seus respectivos conjuntos de respostas. Tais sites estão conceitualizados na literatura de duas maneiras distintas: \textit{community Q\&A} [REF] e \textit{social Q\&A} [REF].

% O primeiro termo supracitado é mais específico do que o segundo: para um site ser classificado assim, é preciso que haja uma identificação de indicadores formais de comunidade, como usuários se mostrando engajados em divulgá-lo, adotando e expressando uma identidade, etc. [REF]

% \textit{Social Q\&A}, de acordo com [REF], é um termo mais abragente e se refere aos sites nos quais os respectivos usuários realizam perguntas, respondem a perguntas e avaliam o conteúdo do site enquanto estão interagindo com ele.

% Apesar de sites do tipo \textit{social Q\&A} serem considerados instâncias de comunidades \textit{online} [REF], o que pode remeter ao conceito de \textit{community Q\&A}, a maior parte dos trabalhos utilizados na elaboração da fundamentação teórica desta pesquisa e, portanto, relevantes neste escopo, opta por utilizar o primeiro conceito definido anteriormente como \textit{social Q\&A}. Tal opção também foi escolhida durante a elaboração desta dissertação.

% O site \textit{Stack Overflow}\footnote{http://stackoverflow.com/} é um exemplo do tipo destacado acima. Tal ambiente colaborativo foi concebido em 2008 e é creditado como o maior site de perguntas e respostas voltados para programadores que desejam tirar dúvidas e oferecer ajuda [REF]. Possui aproximadamente dois milhões de contribuidores [REF] e costuma ter suas perguntas respondidas em um tempo mediano de 11 minutos [REF].

% Em 2014, foi criada uma versão em português do site supracitado chamada \textit{Stack Overflow em Português}, com pouco mais de 14 mil usuários registrados [REF]. Tal site foi utilizado como estudo de caso nesta pesquisa (isto está exposto em detalhes mais adiante nesta dissertação).

% COLOCAR AQUI EXEMPLOS COM FIGURAS DO MOTIVO PELO QUAL O SO PT É CONSIDERADO DO TIPO SOCIAL Q\&A

\begin{itemize}
  \item Definiu-se para o escopo detste trabalho o conceito de sites \qa, que são aqueles os quais são acessados por usuários com o intuito de: realizar perguntas, responder a perguntas ou, ainda, visualizar as discussões geradas por perguntas e os seus respectivos conjuntos de respostas.
  \item Os sites \textit{Q\&A} são classificados na literatura de duas maneiras distintas: \textit{community Q\&A} e \textit{social Q\&A} \cite{gazan2011social};
  \item O que os dois conceitos querem dizer exatamente? Qual é a diferença entre os dois conceitos acima? (R: está constatato na literatura que o termo \textit{community Q\&A} é usado muitas vezes de forma indiscriminada \cite{rosenbaum2010structuration}... acontece que, para usar tal termo, é preciso haver indicadores formais de comunidade \cite{kling2005understanding}) Qual é o mais utilizado pelas minhas referências? Qual é o que mais se encaixa no meu escopo? Por quê?
  \item Mostrar a anatomia de um site \textit{Q\&A} dando como exemplo o \textit{Stack Overflow em Português}, mostrando figuras para destacar o modelo de funcionamento do site;
  \item Trabalhos relacionados:
    \begin{itemize}
        \item Os 3 campos primários de pesquisa em sites \textit{Q\&A} são classificadas assim: estudos sobre motivações e comportamentos dos usuários, avaliações da qualidade da informação contida nestes sites e análises de fatores tecnológicos que exercem influência na participação de usuários \cite{shah2009research}.
        \item Este trabalho tangencia o campo de avaliações da qualidade da informação, pois eu tentei fazer uma análise quantitativa sobre o efeito do compartilhamento na obtenção ou não de respostas úteis para os autores das perguntas compartilhadas;
        \item Este trabalho está, definitivamente, mais em sincronia com o campo de estudos sobre motivações e comportamentos dos usuários, pois aqui eu investiguei, durante a maior parte do trabalho, como os usuários se comportam em relação ao compartilhamento de perguntas e que tipo de design pode ou não interferir neste processo. 
    \end{itemize}
\end{itemize}
\section{\textit{Friendsourcing}}
\begin{itemize}
    \item Definir \textit{crowdsourcing} \cite{brabham2008crowdsourcing};
    \item Definir \textit{friendsourcing} como uma forma de \textit{crowdsourcing} que visa coletar informação disponível em grupos pequenos e socialmente conectados de forma precisa (como, por exemplo, perguntar algo aos amigos) \cite{Bernstein:2008:PVF:1746259.1746260};
    \item Trabalhos relacionados: estudos que envolvem experimentar o uso de tal técnica em certos contextos são os que se parecem com o meu trabalho;
    \item O meu trabalho visa estudar \textit{friendsourcing} em sites \textit{Q\&A} de uma forma que nem sempre quem tenta obter informação dos amigos está tentando ajudar a si mesmo; 
  \end{itemize}
\subsection{Capital Social}
    \begin{itemize}
        \item Definir capital social \cite{portes2000social};
        \item Mostrar que este conceito está presente na prática de \textit{friendsourcing};
        \item Mostrar como capital social aparece no meu trabalho.
    \end{itemize}