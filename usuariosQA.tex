\chapter{Compartilhamento de Perguntas em Sites \qa}
Como forma de entender como os usuários de sites \qa se comportam em relação ao compartilhamento de perguntas de tais sites em redes sociais, foi realizada uma investigação que nos levou a entender melhor esta questão. Os detalhes deste estudo e da discussão gerada por ele, que ajudou a guiar o restante desta pesquisa de mestrado, encontram-se neste capítulo.
\section{Motivação}
Nos capítulos anteriores deste documento está claro que o intuito desta pesquisa de mestrado era explorar o compartilhamento de perguntas de sites \qa em redes sociais. Tal exploração consistiu, primeiramente, num entendimento sobre como os usuários utilizam o compartilhamento de perguntas destes sites nas redes sociais. O entendimento desta questão seria útil para que alguma possível melhoria neste âmbito pudesse ser investigada mais a fundo.

Nosso intuito era ter uma impressão sobre como os usuários se comportam diante da possibilidade de compartilhar perguntas de sites \qa em redes sociais. Queríamos saber o quão comum é este comportamento e como os usuários decidem compartilhar ou não perguntas de sites \qanospace.

\section{Metodologia}
Optamos por realizar um estudo qualitativo porque estávamos interessados em descobrir, no nosso contexto, como os usuários tomam decisões e como eles se comportam. Sendo assim, nosso intuito era coletar relatos de pessoas, o que tornou evidente a necessidade de realizar uma pesquisa qualitativa em forma de entrevistas semiestruturadas \cite{hennink2010qualitative} para que pudéssemos coletar os dados desejados. 

Um guia de entrevista inicial (ver Apêndice A) foi elaborado com os seguintes objetivos:
\begin{itemize}
\item Entender como os usuários tomam a decisão de compartilhar nas redes sociais uma questão originalmente realizada em algum site \qa por outra pessoa;
\item Elencar as motivações associadas a este comportamento;
\item Identificar fatores que afetam as motivações supracitadas.
\end{itemize}

Foram conduzidas, ao todo, doze entrevistas. Os entrevistados foram recrutados de duas maneiras distintas. A maior parte era formada por colegas dos pesquisadores envolvidos neste trabalho e que têm o hábito de compartilhar conteúdo de sites \qa em redes sociais. O resto dos participantes foi convidado para participar desta investigação depois de uma consulta na base de dados do site \textit{Stack Overflow em Português}. À época da investigação, apenas cinco usuários deste site tinham medalhas relativas ao compartilhamento de perguntas em redes sociais, como foi ilustrado no capítulo anterior deste documento (ver Figura ~\ref{fig:medalhassopt}). Destes, apenas três concordaram em participar da investigação em questão, por isto a necessidade de recrutar participantes de outra maneira.

Os entrevistados responderam perguntas sobre: as motivações que eles tinham para compartilhar perguntas, assuntos que os interessavam, utilização da Internet para perguntar e responder, uso das redes sociais para compartilhar conteúdo, custos sociais percebidos por eles em todos estes contextos, etc.

Após cada entrevista, as respectivas respostas foram transcritas. Tais transcrições foram utilizadas para a construção de códigos e, posteriormente, mapas de conceitos que eram constituídos das relações entre tais códigos. Cada código representava um conceito extraído das respostas coletadas (mais detalhes adiante neste capítulo). Tal fluxo de trabalho está de acordo com o que foi descrito com Miles et al. \cite{miles2013qualitative} como sendo um método de análise de dados qualitativos oriundos de entrevistas. 

\section{Análise dos Relatos Obtidos}
\textit{Quora}, \textit{Stack Overflow} e \textit{Stack Overflow em Português} foram os sites mencionados pelos entrevistados. As redes sociais \textit{Facebook} e \textit{Twitter} também apareceram nos relatos. Os assuntos que tipicamente apareceram nas perguntas dos sites \qa e que foram compartilhadas em redes sociais pelos participantes desta investigação foram separadas em duas categorias distintas. Esta classificação foi utilizada para a geração de códigos a partir dos dados coletados. As categorias são:
\begin{itemize}
    \item Perguntas profissionais --- qualquer pergunta cujo assunto esteja relacionado com alguma obrigação profissional do usuário em questão foi classificada desta maneira;
    \item Perguntas sobre \textit{hobby} --- qualquer pergunta cujo assunto esteja relacionado com alguma preferência pessoal do usuário e que não tenha a ver com nenhuma obrigação profissional foi classificada desta maneira.
\end{itemize}

Quando foi possível identificar saturação nos dados, ou seja, quando as respostas começaram a se repetir e os pesquisadores perceberam que não havia mais a necessidade de se realizar novos questionamentos nas entrevistas, o conjunto final de códigos foi analisado pela última vez. Tal análise fez emergir, finalmente, os conceitos mais importantes que os dados estavam evidenciando. A lista abaixo contém os conceitos mais importantes que foram elencados nesta análise, bem como exemplos de trechos dos relatos que ilustram os mesmos: 
\begin{itemize}
    \item Conceito: motivação. Exemplos: ``porque alguns colegas meus estavam interessados na resposta, aí achei legal compartilhar'' e ``apenas porque eu queria aumentar as minhas chances de obter alguma resposta'';
    \item Conceito: compartilhamento de perguntas. Exemplos: ``eu vi a pergunta no Quora e decidi compartilhar ela [\textit{sic}]''  e ``eu tava [\textit{sic}] precisando resolver um problema envolvendo ponteiros aí encontrei uma pergunta com uma discussão muito útil, então compartilhei porque algum amigo meu poderia gostar da discussão'';
    \item Conceito: identidade. Exemplos: ``assim como eu, tenho alguns amigos que gostam de ler sobre futebol na Internet'' e ``passo horas discutindo sobre feminismo com as pessoas no Facebook porque tenho amigos que também se interessam'';
    \item Conceito: interesse. Exemplos: ``este é um assunto interessante, na minha opinião'', ``eu tou [\textit{sic}] muito interessado em coisas sobre o feminismo'';
    \item Conceito: profissional. Exemplos: ``é que eu sou engenheiro de software, trabalho com aplicativos para celular, aí precisei tirar uma dúvida sobre iOS'', ``eu estava fazendo um site em Django para a um cliente, aí passei por um probleminha'';
    \item Conceito: \textit{hobby}. Exemplos: ``eu sou cientista da computação, mas nas horas vagas eu gosto de ler sobre futebol e música mesmo'', ``acho que quando não tou [\textit{sic}] trabalhando eu fico jogando e lendo sobre games a maior parte do tempo''.
\end{itemize}

Os participantes relataram como eles decidem compartilhar ou não alguma pergunta. O comportamento de ajudar autores de perguntas compartilhando-as foi pouco relatado. Segundo os entrevistados, a decisão de se compartilhar ou não uma pergunta de um site \qa nas redes sociais é feita baseada em alguns critérios que estão citados no que se segue.

Esperava-se encontrar, nos relatos, indicativos de que os usuários compartilhavam perguntas de sites \qa nas suas redes sociais com o intuito de oferecer ajuda aos donos das perguntas (divulgando a pergunta para que alguém, que soubesse, pudesse ver). Entretanto, este comportamento social, de acordo com as análises, foi considerado um tanto quanto raro. Ao invés disto, os entrevistados relataram que estavam mais interessados em ajudar a si mesmos quando compartilharam alguma pergunta de algum site \qa que estava sem resposta.

O \textit{Facebook} apareceu, nos dados, como a rede social mais popular porque ``todo mundo está no Facebook'', como disse um dos entrevistados. Relacionamentos interpessoais foram apontados como os motivos mais importantes para os usuários utilizarem o \textit{Facebook}. Sendo assim, qualquer ação desta rede social deve contribuir, de alguma forma, para a construção ou manutenção de laços sociais. Isto vale, então, para o compartilhamento de perguntas oriundas de sites \qa, fazendo com que seja necessário mostrar aos usuários quais são os ganhos sociais, se existirem, associados a esta prática se o intuito for promovê-la. 

Foi possível entender, de acordo com os relatos obtidos, que um dado usuário de uma rede social que publica ou compartilha conteúdo em excesso tende a perder capital social, tendo em visto que este tipo de comportamento geralmente incomoda as pessoas. Além disto, como os entrevistados informaram, os custos sociais associados com o compartilhamento de perguntas realizadas por outras pessoas e sem respostas é bastante elevado. Por outro lado, os custos relacionados com o compartilhamento de alguma pergunta já respondida cujo assunto seja de interesse do próprio usuário e de seus amigos é aceitável, ainda mais se isto vier com um conjunto de respostas.

No caso das perguntas originalmente publicadas em algum site \qa e que estava sem resposta, a maioria dos usuários nos informou que eles realizam o compartilhamento neste caso, geralmente, quando eles mesmos são os autores das perguntas. O motivo para este comportamento é óbvio: ajudar a si mesmo aumento as chances de ter a pergunta respondida. Se a pergunta, neste âmbito, for realizada por uma terceira pessoa, é muito mais provável que outros usuários não a compartilhem. É interessante observar este tipo de egoísmo. Certamente não está claro para estes usuários quais são os ganhos sociais relacionados com esta prática. Este aspecto deve ser considerado como um desafio para promover este comportamento social de compartilhar, em redes sociais, perguntas de sites \qa realizadas por terceiros.

Um detalhe que certamente diminui os custos sociais relacionados com o compartilhamento de uma pergunta de um site \qa é o fato da pergunta ter o potencial de gerar uma discussão interessante do ponto de vista de quem compartilhou e dos seus amigos. Isto acontece porque os usuários das redes sociais estão interessados em reforçar laços sociais, como foi mostrado em um trabalho anterior por Jiang et al. \cite{jiang2009social} Além disto, se a pergunta se enquadrar na categoria \textit{hobby}, os custos sociais ficam menores ainda, pois é mais aceitável compartilhar conteúdo não-profissional nas redes sociais, de acordo com os usuários entrevistados.

Sobre as perguntas que se encaixam na categoria profissional, foi possível perceber que os usuários aceitam melhor arcar com os custos sociais relacionados com o compartilhamento deste conteúdo se a pergunta já tiver respondida e se eles já souberem, de antemão, que pelo menos algum amigo tem interesse no tópico em questão. Esta percepção sugere, também que os ganhos sociais relacionados com o compartilhamento de perguntas de cunho profissional e sem respostas não estão claros para os usuários.

Alguns chegaram a comentar que, como os sites \qa em questão não oferecem a funcionalidade de selecionar respostas das redes sociais para as perguntas, a atitude de compartilhar tais perguntas nas redes sociais perde um pouco o sentido. Um dos entrevistados disse: ``acho muito nada a ver compartilhar pergunta do Stack Overflow, por exemplo, no Facebook... não tem nem como selecionar a resposta de alguém que por acaso queira responder no próprio Facebook''. Sem dúvida esta impressão de alguns usuários é um dos desafios a serem enfrentados pelos sites \qa que desejam promover tal prática.

Finalmente, de maneira geral, nossos resultados qualitativos sugerem que compartilhar perguntas sem respostas de sites \qa nas redes sociais não é uma prática comum. Isto se deve, principalmente, ao fato de que, tipicamente, não está claro para os usuários se eles poderão ganhar algum capital social agindo assim, fazendo com que o custo associado com esta prática seja elevado. Vale ressaltar que os três sites \qa mencionados pelos entrevistados, \textit{Quora}, \textit{Stack Overflow} e \textit{Stack Overflow em Português}, tentam promover o compartilhamento de perguntas com a seguinte mensagem simples no botão de compartilhar: ``compartilhar''.