Sites \qa são aqueles acessados por usuários com o intuito de: encontrar uma resposta para alguma pergunta ou ajudar alguém a conseguir isto. Tais sites são muito utilizados, por exemplo, por programadores que desejam esclarecer dúvidas gerais sobre programação, como é o caso do site \textit{Stack Overflow em Português}. Muitos pesquisadores têm concentrado esforços visando descobrir maneiras de atrair mais conteúdo para sites deste tipo e, também, entender comportamentos e motivações dos respectivos usuários. Investigou-se, neste trabalho, a utilização da prática do compartilhamento de perguntas oriundas de sites \qa em redes sociais como estratégia para fomentar a contribuição em tais sites. Nosso intuito era: entender como os usuários se comportam diante de tal funcionalidade, investigar como promover tal prática e, por fim, tentar encontrar indícios acerca da utilidade desta prática como meio de fomentar a contribuição em sites \qanospace. Foram feitas duas investigações qualitativas nas quais usuários de sites \qa e de redes sociais foram entrevistados e relataram como se comportam diante do compartilhamento de perguntas. Posteriormente, foi realizado um experimento quantitativo para testar diferentes formas de incentivar o compartilhamento de perguntas oriundas de sites \qa em redes sociais, bem como para investigar mais a fundo o efeito desta prática no conteúdo destes sites. Descobriu-se que os usuários são muito criteriosos na escolha dos conteúdos que eles compartilham nas suas redes sociais por causa dos custos sociais envolvidos. Por este motivo concluímos que é preciso repensar a forma com a qual se tenta promover o compartilhamento de perguntas de tais sites em redes sociais. Além disto, também foi possível perceber que tal compartilhamento pode não ser uma boa escolha para quem pretende fomentar a contribuição nestes sites em questão.