\chapter{Protocolo final de entrevista do primeiro estudo qualitativo}
O que se segue é a lista final de perguntas pré-estabelecidas utilizadas no primeiro estudo qualitativo realizado durante esta pesquisa de mestrado. Eis as perguntas:
\begin{itemize}
    \item Quantos anos você tem?
    \item Qual é o seu grau de escolaridade?
    \item Você permanece conectado à Internet, diariamente e em média, durante quantas horas?
    \item Você acessa a Internet por meio de qual(quais) dispositivo(s)?
    \item Você utiliza a Internet, tipicamente e nas horas de lazer, para fazer o quê?
    \item Você utiliza a Internet, tipicamente e durante o trabalho/estudo, para fazer o quê?
    \item Você costuma utilizar as suas redes sociais para fazer o quê?
    \item Qual é a rede social que você mais utiliza? Por quê?
    \item Como você utiliza a Internet para perguntar coisas? 
    \item Como você utiliza a Internet para responder coisas?
    \item Você poderia dizer como foi a sua experiência ao compartilhar, em alguma rede social, uma pergunta feita por outrem? Por que você fez isto?
    \item Você acredita que as outras pessoas costumam fazer isto também? Por quê?
    \item Você enxerga algum custo relacionado a esta atitude? Qual? Em quais situações você acha que este custo faria com que você não compartilhasse a(s) pergunta(s)?
    \item Como você descreveria o(s) assunto(s) e o(s) tipo(s) da(s) pergunta(s) feitas por outra(s) pessoa(s) que você compartilhou em alguma rede social sua?
    \item Você acredita que a sua atitude ajudou, de alguma forma, o dono da pergunta? Por quê? Você acha que alguma informação citada na discussão gerada pelo seu compartilhamento poderia ser útil para o dono da pergunta?
    \item Você já teve a chance de repetir esta atitude em uma outra oportunidade e não o fez? Por quê?
    \item Como você enxerga a influência da relação que você tem com o(s) autor(es) da(s) pergunta(s) na sua motivação para compartilhar?
    \item Como você enxerga a relação do(s) tipo(s) e do(s) assunto(s) das perguntas com a sua atitude de compartilhar?
    \item Como você enxerga as redes sociais equanto ambientes para se realizar perguntas?
    \item Como você descreveria um ambiente ideal, na Internet, para se perguntar e responder coisas?
\end{itemize}

\chapter{Protocolo final de entrevista do segundo estudo qualitativo}
O que se segue é a lista final de perguntas pré-estabelecidas utilizadas no segundo estudo qualitativo realizado durante esta pesquisa de mestrado. Eis as perguntas:
\begin{itemize}
\item Quantos anos você tem?
    \item Qual é o seu grau de escolaridade?
    \item Você permanece conectado à Internet, diariamente e em média, durante quantas horas?
    \item Você acessa a Internet por meio de qual(quais) dispositivo(s)?
    \item Você utiliza a Internet, tipicamente e nas horas de lazer, para fazer o quê?
    \item Você utiliza a Internet, tipicamente e durante o trabalho/estudo, para fazer o quê?
    \item Você costuma utilizar as suas redes sociais para fazer o quê?
    \item Qual é a rede social que você mais utiliza? Por quê?
    \item Como você utiliza a Internet para perguntar coisas? 
    \item Como você utiliza a Internet para responder coisas?
    \item Você poderia dizer como foi a sua experiência ao direcionar a pergunta de alguém para alguma outra pessoa em grupos do Facebook?
    \item Por que você fez isto?
    \item Na sua opinião, o que você ganhou com isto?
    \item Você tinha alguma relação com o dono original da pergunta?
    \item Você já viu mais alguém fazer isto? Se sim, você acha que deu certo? Você acha que a pessoa que fez isto ganhou o quê?
    \item Você acha que as pessoas costumam fazer isto?
    \item Você já desejou que alguém fizesse isto com alguma pergunta sua? Já fizeram? Se fizeram, esta atitude lhe ajudou em algo? Quem fez isto?
    \item Por que você utiliza o Facebook?
    \item Por que você participa deste(s) grupo(s) de perguntas e respostas no Facebook?
\end{itemize}