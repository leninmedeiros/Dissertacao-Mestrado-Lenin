\chapter{Introdução}
\begin{itemize}
    \item Sites de q\&a são alvos de vários estudos sobre obter/melhorar respostas e achar \textit{experts}, como REF1, REF2, etc.;
    \item Além disso, a forma como os usuários interagem com tais sites também é investigada na literatura, como REF1, REF2, etc.;
    \item (Falar do Forrósquare aqui->) Dado um esforço preliminar para criar, manter e observar in loco um site de q\&a com o intuito de se elencar novos possíveis estudos a serem realizados, identificou-se a necessidade de abordar o uso do compartilhamento de perguntas em redes sociais para ajudar a obter respostas;
    \item Explicar bem resumidamente o que foi feito a partir daí.
\end{itemize}

\section{Motivação}
\begin{itemize}
\item As redes sociais estão bastante presentes na Internet e praticamente todos os sites possuem algum tipo de integração com elas (CITAR FONTES);
\item Existem esforços na literatura para obter obter/melhorar respostas, etc., como foi dito anteriormente... mas não foi encontrado nenhum esforço para estudar como o compartilhamento em redes sociais pode entrar na jogada.
\end{itemize}
\section{Objetivos}
\begin{itemize}
\item Analisar como as pessoas se comportam ao decidir compartilhar ou não uma pergunta de um site q\&a em alguma rede social;
\item Inferir possíveis cenários que podem influenciar esta decisão;
\item Explorar tais cenários com o intuito de identificar os efeitos dos mesmos, se houver, na decisão de compartilhar ou não uma pergunta em alguma rede social.
\end{itemize}
\section{Contribuições Esperadas}
\begin{itemize}
\item Uma análise do comportamento dos usuários de sites q\&a em relação ao compartilhamento;
\item Uma análise sobre a influência do compartilhamento de perguntas na obtenção, ou não, de respostas em sites q\&a;
\item Uma análise sobre a interferência de certos cenários na tentativa de promover o compartilhamento de perguntas em sites de q\&a e as influências dos compartilhamentos nos cenários na obtençãom, ou não, de respostas.
\end{itemize}

\section{Estrutura do Documento}
Colocar aqui a estrutura deste documento (o que o leitor vai encontrar em cada parte do texto).