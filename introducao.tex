\chapter{Introdução}
    \begin{itemize}
        \item LEMBRAR DE DEFINIR O QUE SÃO REDES SOCIAIS AQUI NA INTRODUÇÃO;
        \item O que são sites de Q\&A. Exemplificar.
        \item Um Site que se sobresai é o SO. Exibir números para enfatizar a grandeza e importância do \textit{Stack Overflow}: quantidade de usuários, volume de conteúdo gerado, tempo médio de obtenção de resposta para as perguntas feitas no site, etc.; 
        % O site \textit{Stack Overflow}\footnote{http://stackoverflow.com/} é um exemplo do tipo destacado acima. Tal ambiente colaborativo foi concebido em 2008 e é creditado como o maior site de perguntas e respostas voltados para programadores que desejam tirar dúvidas e oferecer ajuda [REF]. Possui aproximadamente dois milhões de contribuidores [REF] e costuma ter suas perguntas respondidas em um tempo mediano de 11 minutos [REF].

% Em 2014, foi criada uma versão em português do site supracitado chamada \textit{Stack Overflow em Português}, com pouco mais de 14 mil usuários registrados [REF]. Tal site foi utilizado como estudo de caso nesta pesquisa (isto está exposto em detalhes mais adiante nesta dissertação).
        \item Promover contribuições é uma parte fundamental de operar sites deste tipo.0
        \item Visão geral de como tais sites estão estudados na literatura quanto a formentar contribuições;
        \item A motivação deste estudo parte de uma experiência prática com a operação de um site de QA,
        \item Descrever Forrósquare aqui, mencionar que esbarramos em dificuldades de fomentar o uso do sistema, e que percebemos a oportunidade de usar redes sociais já estabelecidas como forma de alimentar o site. (Nao alongar essa parte. Ela serve para inspirar a ideia de usar outras redes, e depois ligamos dizendo que voltando para a literatura, não encontramos estudos sobre isso).
            \begin{itemize}
                \item O que era o projeto Forrósquare?
                \item Quando, onde e como aconteceu?
                \item Qual era o intuito?
                \item O que se descobriu com os dados gerados pelo Forrósquare?
                \item Explicar que, graças ao Forrósquare, os pesquisadores se perguntaram a relação entre compartilhamento de perguntas em redes sociais e os sites \textit{Q\&A} (explicar como se chegou a tal questionamento);
                \item Redes sociais podem ser descritas como serviços da web nos quais é permitido a um dado usuário: criar um perfil público ou semi-público dentro de algum escopo limitado, articular alguma lista contendo outros usuários do mesmo serviço com os quais ele possui algum tipo de conexão e, ainda, visualizar e percorrer esta lista de conexões, bem como as listas de conexões dos outros usuários \cite{ellison2007social}. O objeto de estudo deste trabalho é um site \qa que não se comporta como uma rede social, tendo em vista que não atende aos 3 requisitos listados anteriormente. Vale ressaltar que é possível que algum site desta natureza também seja uma rede social. 
            \end{itemize}
        \item Explicar, resumidamente, o que foi feito a partir do Forrósquare e os resultados obtidos:
            \begin{itemize}
                \item Entendi como funciona o compartilhamento de perguntas de sites \textit{Q\&A} do ponto de vista dos usuários;
                \item Vi o que os dados quantitativos reais do Stack Overflow me disseram sobre a questão do compartilhamento;
                \item A partir do estudo qualitativo, defini 3 propostas de \textit{design} e vi como os usuários se comportaram em relação a elas por meio do umaforcaso.pt;
                \item Vi o que os dados quantitativos reais do umaforcaso.pt me disseram sobre a questão do compartilhamento;
                \item Minhas conclusões gerais do trabalho.
            \end{itemize}
\end{itemize}

\section{Motivação}

DO ARTIGO SOBRE O ESTADO DA ARTE DE SQA: It is becoming less and less accurate to think of SQA sites in a vacuum. Social networking sites such as Facebook allow the sharing of questions users find interesting on any SQA site, and it is possible to rate and interact with the same SQA content across multiple platforms.

    Discutir aqui a motivação deste trabalho:
    \begin{itemize}
        \item Investigar o que pode trazer mais conteúdo satisfatório (resposta para perguntas) é algo bem presente na literatura. Então, sempre existe demanda para investigar novas maneiras de fazer isto. Aqui eu vou investigar se o compartilhamento de perguntas pode ajudar, neste sentido, o Stack Overflow. Sendo assim, esta é uma motivação;
        \item Mostrar que as redes sociais estão bastante presentes na Internet e praticamente todos os sites possuem algum tipo de integração com elas (o que vai ser uma deixa para o item abaixo);
        \item Uma das integrações de sites \textit{Q\&A} com as redes sociais é o botão de compartilhar, mas só existem esforços na literatura para obter obter/melhorar respostas, traçar perfis de usuários, elencar tipo de conteúdo gerado, etc. O fato de não haver nenhum estudo sobre isto na literatura é outra motivação.
    \end{itemize}
    \section{Objetivos}
\begin{itemize}
    \item Objetivo geral: estudar se e como o compartilhamento de conteúdo em redes sociais fomenta a contribuição em sites \textit{Q\&A};
    \item Objetivos específicos:
    \begin{itemize}
        \item Entender como usuários de sites \textit{Q\&A} percebem as vantagens e custos de compartilhar conteúdo em redes sociais, e se e como o fazem.
        % se comportam ao compartilhar, nas suas redes sociais, perguntas destes sites. As seguintes perguntas deverão ser respondidas por este estudo (tais perguntas deverão estar aqui, mas as respectivas respostas deverão aparecer mais adiante na dissertação):
        %     \begin{itemize}
        %         \item Este comportamento de compartilhar é comum? Por quê?
        %         \item Como os usuários decidem se devem ou não compartilhar uma pergunta?
        %         \item Que tipo de pergunta de \textit{sites Q\&A}, tipicamente, se compartilha em redes sociais?
        %         \item Qual o intuito de se realizar tal compartilhamento?
        %         \item Na visão dos usuários, compartilhar faz alguma diferença positiva ou negativa?
        %     \end{itemize}
        \item Investigar deficiências e oportuniades de melhoria nos mecanismos atuais de compartilhamento de sqa.
        \item Experimentar com alternativas de design em um estudo de caso para fomentar respostas em um sqa através do compartilhamento em redes sociais.
        
    \end{itemize}
\end{itemize}

\section{Estrutura do Documento}
    \begin{itemize}
        \item Colocar aqui como o documento está estruturado e o que o leitor deverá encontrar em cada parte. COlocar o que cada capítulo tem a ver com os objetivos...
    \end{itemize}