\chapter{Compartilhamento de Perguntas em Grupos no \textit{Facebook}}

% Breve resumo do que foi feito aqui. Como tudo aqui é bem parecido com o capítulo anterior, então deve-se ter em mente que eu não preciso inventar a roda duas vezes: o que já foi explicado não precisa de uma nova explicação, precisa apenas ser devidamente referenciado.
% \section{Motivação}
% \begin{itemize}
% \item Por que eu precisei fazer este experimento? 
% \item O que eu queria investigar?
% \item O que eu esperava encontrar?
% \end{itemize}
% \section{Metodologia}
% \begin{itemize}
% \item Por que este experimento teve que ser qualitativo?
% \item Qual foi o método qualitativo empregado e por que ele foi escolhido? Explicar em detalhes o método e explicar também o motivo pelo qual eu escolhi fazer entrevistas abertas (explicando o que são estas tais entrevistas abertas);
% \item Como foram as 8 entrevistas? Quem eram os participantes?
% \end{itemize}
% \section{Análise}
% \begin{itemize}
% \item Como foi o processo de codificação?
% \item Como está disposto o conjunto final de códigos?
% \end{itemize}
% \section{Resultados}
% \begin{itemize}
% \item O que o conjunto final de códigos me disse?
% \item Qual é a ligação entre tais resultados e o que existe na literatura?
% \item Qual é a ligação entre tais resultados e o que eu esperava encontrar?
% \end{itemize}