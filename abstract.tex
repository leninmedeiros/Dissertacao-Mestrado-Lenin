\qa sites are those which are accessed by users who want to find some answer to some question or help someone else to get this. These websites are very used, for instance, by programmers in order to clarify general questions about programming. This behavior could be found in a website called \textit{Stack Overflow em Português}. Many researchers have been focused on finding out ways to attract more content to these websites and, also, on the aspects of behaviors and motivations of its users. In this work our focus was the use of the practice of sharing \qa sites questions in social networks as strategy to increase the content of these websites. Our aim was: to extract some lessons about how is the users' behavior when they face the functionality of to share questions in \qa sites, investigate how to promote this behavior and find out evidences about the advantages for \qa sites related to this practice as way to increase the contribution to this kind of websites. We conducted two qualitative investigations in which users from \qa sites and social networks were interviewed and they related how were they behaviors when they faced the sharing of questions. Then, we made a quantitative experiment in order to test different ways of promoting the sharing of \qa sites questions in social networks, as well as to investigate more deeply the effect of this practice in these websites contents. We found that the users are very discerning when they decide to share or not content in their social networks because of the related social costs. That's why we believe that it's necessary to change the way to promote the sharing of \qa sites questions in social networks. Besides that, it's also possible to notice that this sharing of questions may not be a good strategy to increase the contribution in these websites.  